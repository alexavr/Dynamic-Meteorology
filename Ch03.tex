\chapter{Уравнения гидродинамики и термодинамики} \label{ch03}

\section{{\color{noone} Системы координат, используемые в метеорологии}}

\section{{\color{noone} Уравнения движения в форме Лагранжа и в форме Эйлера}}

\section{{\color{noone} Изменения гидростатического давления}}

\section{{\color{noone} Связь между статическим и гидродинамическим давлением}}

\section{{\color{noone} Сила градиента давления}}

\section{{\color{noone} Сила гравитации}}

\section{{\color{vms}Уравнения движения в инерциальной системе координат}}
    Напомним, что существуют инерциальные системы координат, в которых тело, на которое оказывают влияние силы, движется прямолинейно и равномерно (1 закон Ньютона). Если система координат движется относительно инерциальной системы с постоянной скоростью, то она тоже является инерциальной. Если же система координат движется относительно инерциальной системы координат с ускорением, то она является неинерциальной. То в ней тело, на которые не оказывают влияние никакие силы, получает дополнительное ускорение, связанное с самой системой координат. 
        
    Поскольку мы находимся на вращающейся планете, естественно связать нашу систему координат с точкой на поверхности Земли. Такая система координат называется локальной и будет являться неинерциальной, поскольку закреплена на поверхности выражающейся Земли, а вращение -- это движение с ускорением. В ней движение частицы воздуха будет получать дополнительное ускорение, которое не связано с действием каких-либо физических сил. 
    
\section{{\color{vms} Преобразования операторов при переходе от абсолютной (инерциальной) к вращающейся системе координат}}
    
    Введем преобразование операторов для перехода из неинерциальной системы координат в инерциальную. Положим, что векторы $i$, $j$ и $k$ -- базис векторы нашей системы. Эта система вращается с угловой скоростью $\Omega$ против часовой стрелки вокруг оси вращения и направлена параллельно этой оси. Величина угловой скорости будет:
    \begin{equation}
        \Omega = \frac{2\pi}{T} \cong 7.3\cdot10^{-5} c^{-1},
    \end{equation}
    где $T$ -- время полного оборота планеты (допустим 86400 секунд). Следовательно, базис-векторы нашей неинерциальной системы координат тоже будут вращаться с этой угловой скоростью. Тогда 
    \begin{equation}
    \label{omega}
        \td{\vec{i}}{t}=(\vec{\Omega}\times\vec{i}); \
        \td{\vec{j}}{t}=(\vec{\Omega}\times\vec{j}); \
        \td{\vec{k}}{t}=(\vec{\Omega}\times\vec{k}).   
    \end{equation}
    Рассмотрим произвольный вектор $\vec{A}$, который может характеризовать какое-то физическое поле (скорость, градиент давления и т.д.). Его можно разложить в двух системах координат: в локальной декартовой
    \begin{equation}
        \vec{A} = A_x\vec{i}+A_y\vec{j}+A_z\vec{k}. 
    \end{equation}
    и в инерциальной системе (например той, в которой оси направлены на небесные светила) и обозначим ее индексом $a$
    \begin{equation}
        \vec{A} = A_x^a\vec{i^a}+A_y^a\vec{j^a}+A_z^a\vec{k^a}.
    \end{equation}
    Возьмем производную по времени от этих векторов, учитывая, что базис векторы неинерциальной системы координат тоже изменяются во времени:
    \begin{equation}
    \begin{aligned}
    \label{dadt1}
        \td{\vec{A}}{t} &=\td{A_x^a}{t}\vec{i^a} + \td{A_y^a}{t}\vec{j^a} + \td{A_z^a}{t}\vec{k^a} = \\ 
        &= 
            \td{A_x}{t}\vec{i} 
            + \td{A_y}{t}\vec{j} 
            + \td{A_z}{t}\vec{k}
            + A_x\td{\vec{i}}{t}  
            + A_y\td{\vec{j}}{t}  
            + A_z\td{\vec{k}}{t}. 
    \end{aligned}
    \end{equation}
    Первые три слагаемые в правой части уравнения (\ref{dadt1}) являются полной производной вектора $\vec{A}$ в неинерциальной (вращающейся) системе координат, а последние три слагаемых, используя равенство \ref{omega}, можно переписать как твердотельное вращение вектора $\vec{A}$. Тогда 
    \begin{equation}
    \label{dadt2}
        \frac{d_a\vec{A}}{t} = \td{\vec{A}}{t} + (\vec{\Omega}\times\vec{A}),
    \end{equation}
    где индекс $a$ обозначает абсолютную (инерциальную) систему координат. Таким образом, скорость измерения вектора $\vec{A}$ в абсолютной системе координат равна скорости изменения этого же вектора во вращающейся системе плюс и самого вращения. 

\section{{\color{noone} Сила Кориолиса и центробежная сила}}


    

\section{{\color{noone} Уравнения движения во вращающейся системе координат}}
    
    % Чтобы задать вращение этой системы 

    % Рассмотрим для простоты сначала декартову систему координат (\ref{fig3.8.1}). Пусть оси $x$ и $y$ лежат в экваториальной плоскости и вращаются вместе с Землей. Такая система координат будет неинерциальной, поскольку вращение -- это движение с ускорением. D yb
    % \begin{boxAA}
    %     \label{fig3.8.1}
    %     Картинка из видео 2021-10-18 1:00:00.
    % \end{boxAA}

\section{{\color{noone} Уравнения движения в локальной декартовой системе координат}}

\section{Уравнение сохранения массы или уравнение неразрывности}

\section{Сила трения}

\section{Уравнения движения в вязкой жидкости (уравнения Навье-Стокса)}

\section{Уравнение состояния сухого воздуха}

\section{Уравнение притока тепла}

\section{Система уравнений гидротермодинамики в локальных декартовых координатах}

\section{Уравнения гидротермодинамики для турбулентной атмосферы} \label{ch:turb_equations}

\begin{equation}
\label{mom_conserv_turb_u}
    \td{u}{t} = -\frac{1}{\rho}\pd{p}{x} - 2\Omega_yw + 2\Omega_zv + 
    \frac{\partial}{\partial x} \left( k_h \pd{u}{x} \right) + 
    \frac{\partial}{\partial y} \left( k_h \pd{u}{y} \right) + 
    \frac{\partial}{\partial z} \left( k_z \pd{u}{z} \right)
\end{equation}

\begin{equation}
\label{mom_conserv_turb_v}
    \td{v}{t} = -\frac{1}{\rho}\pd{p}{y} - 2\Omega_zu +  
    \frac{\partial}{\partial x} \left( k_h \pd{v}{x} \right) + 
    \frac{\partial}{\partial y} \left( k_h \pd{v}{y} \right) + 
    \frac{\partial}{\partial z} \left( k_z \pd{v}{z} \right)
\end{equation}

\begin{equation}
\label{mom_conserv_turb_w}
    \td{w}{t} = -\frac{1}{\rho}\pd{p}{w} - 2\Omega_yu +  
    \frac{\partial}{\partial x} \left( k_h \pd{w}{x} \right) + 
    \frac{\partial}{\partial y} \left( k_h \pd{w}{y} \right) + 
    \frac{\partial}{\partial z} \left( k_z \pd{w}{z} \right)
\end{equation}

\begin{equation}
\label{mass_conserv_turb}
    \pd{\rho}{t} + \pd{\rho u}{x} + \pd{\rho v}{y} + \pd{\rho w}{z} = 0 
\end{equation}

\begin{equation}
\label{ideal_gas}
    p = \rho TR 
\end{equation}

\begin{multline}
    \label{temp_conserv_turb}
    \td{\Theta}{t} =  
    \frac{\partial}{\partial x} \left( \kappa_h \pd{\Theta}{x} \right) + 
    \frac{\partial}{\partial y} \left( \kappa_h \pd{\Theta}{y} \right) + 
    \frac{\partial}{\partial z} \left( \kappa_z \pd{\Theta}{z} \right) - \\
  - \frac{\Theta}{T} \left( \frac{1}{\rho C_p} \pd{F}{z} \right) + 
    \frac{\Theta}{T} \left( \frac{L}{C_p} \left( C_d + C_e \right) \right)
\end{multline}


\section{{\color{done}Уравнение переноса субстанций}}
При решении многих задач метеорологии  приходится сталкиваться с необходимостью включения в модель примесей: водяного пара, осадков, различных примесей, которые будут иметь, как и осадки, скорость гравитационного оседания. Уравнения переноса являются по своему существу уравнениями сохранения массы и в случае, если примесь является идеальным газом (отсутствует вязкость), то оно в точности совпадает с уравнением неразрывности. В этой связи мы не будем повторять вывод уравнения с самого начала, а остановимся лишь на отличиях уравнения переноса примеси от уравнения неразрывности. Имеются два основных отличия:
\begin{enumerate}
\item Если примесь (осадки, крупные частицы) имеют скорость гравитационного оседания $v_t$, то она вносится как дополнительный компонент в член переноса по вертикали.
\item Вводятся дополнительные члены, описывающее источники (стоки) переносимой субстанции.
\end{enumerate}
Уравнения записываются для турбулентной атмосферы, т.е. вводятся коэффициенты турбулентного обмена также как в и уравнениях движения и притока тепла. С учетом описанных выше особенностей, характерное уравнения переноса примеси имеет следующий вид:
   \begin{multline}
    \label{treq}
        \pd{q}{t} + \pd{(uq)}{x} + \pd{(vq)}{y} + \pd{[(w+v_t)q]}{z} = \\
     = \frac{\partial}{\partial x} \left( k_x \pd{q}{x} \right) + \frac{\partial}{\partial y} \left( k_y \pd{q}{y} \right) + \frac{\partial}{\partial z} \left( k_z \pd{q}{z} \right) + 
     \varepsilon_{\Phi}
    \end{multline}
Здесь $q$ -- отношение смеси любой субстанции, $v_t<0$ -- скорость гравитационного оседания, $\varepsilon_{\Phi}$ -- источники и стоки примеси. В случае влажности и осадков это фазовые переходы влаги в атмосфере, в случае химически активных примесей это различные химические реакции, приводящие к стоку или появлению рассматриваемой примеси. Первых три члена в правой части (\ref{treq}) описывают турбулентный перенос примеси.  Коэффициенты турбулентного обмена ($k_x, k_y, k_z$) по разным направлением считаются известными величинами.
