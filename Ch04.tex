\chapter{Упрощения уравнений гидротермодинамики}
    \begin{warn}
    Сюда нужен вводный текст. Сказать про не решаемость уравнений НС и про необходимость выкручиваться. 
    \end{warn}

\section{Упрощения уравнений на основе масштабного анализа и физических аргументов}

Система (\ref{mom_conserv_turb_u})-(\ref{temp_conserv_turb}), хотя в ней и было проведено "сглаживание" в основном для того, чтобы заменить молекулярную вязкость и теплопроводность на их турбулентные аналоги, отражает движения многий масштабов, начиная от быстрый акустических и гравитационных колебаний и заканчивая медленными волнами синоптического или планетарного масштаба. Поэтому исходная система (\ref{mom_conserv_turb_u})-(\ref{temp_conserv_turb}) как правило упрощается, исходя из того, какие процессы в атмосфере мы хотим изучать, а воспроизведением каких движением мы можем, или даже лучше пожертвовать, чтобы не вносить дополнительный шум. Те или иные упрощения в исходной системе производятся как правило исходя из некоторых общих физических соображений и последующего масштабного анализа или путем оценки порядка величин отдельных членов уравнений, исходя из известных характерных значений переменных, их вертикальных и горизонтальных производных.


Например, при изучении крупномасштабной динамики в свободной атмосфере сразу делается предположение, что силы трения вдали то земной поверхности невелики и уже затем производится дальнейший масштабный анализ для уравнений невязкой жидкости. Напротив, в пограничном слое сразу делается предположение, что силы трения важны, а горизонтальная изменчивость переменных несущественна и производится отбрасывание соответствующих членов.

Следует отметить, что масштабирование применительно к атмосфере проводить не совсем просто, ибо атмосфера Земли, как уже отмечалось ранее, имеет довольно большое отношение аспекта $A=L/h$. Отношением аспекта называют также отношение горизонтального размера той или иной циркуляционной ячейки к ее вертикальному размеру. Например, горизонтальный размер (масштаб) циклона или антициклона $L=10^3$км, а его вертикальная мощность $h = 10$км, то есть $A_{циклона}  \simeq 10^2$. Кучево-дождевые облака имеют примерно одинаковый размер по вертикали и горизонтали, т.е. для него $A \simeq 1$. Этот факт также нужно учитывать при масштабировании и оценке различных членов в уравнениях. В качестве исходных масштабов всегда лучше брать основные: масса $(\tilde{M})$, время $(\tilde{t})$, длина $(L)$. Так как отношение аспекта для атмосферы много больше единицы, то масштаб масштаб длины может быть различным по горизонтали и вертикали и тогда, строго говоря, нужно переходить к вертикальному масштабированию. Однако сейчас мы остановимся на в рамках традиционного для метеорологии квазискалярного масштабирования, подразумевая под масштабом $L$ горизонтальный масштаб возмущений. С определением горизонтального масштаба проблем как правило не возникает, поскольку из опыта мы хорошо знаем горизонтальные масштабы (размеры) различных атмосферных возмущений: циклонов, планетарных волн , конвективных образований. Интерпретация масштаба времени не столь однозначна. Что подразумевать под масштабом времени Имеются различные возможности. Под масштабом времени можно  подразумевать характерное время жизни того или иного атмосферного образования. Например, характерное время жизни циклона составляет 4-5 дней. Если примем характерное время жизни 4 дня, то это составит $3600\times24\times4\simeq3.6\times10^5$с. 



\section{{\color{noone}Уравнения движения на бета-плоскости}}
    \lipsum[1-2]

\section{Некоторые стационарные формы течений}
    \lipsum[1-2]

\subsection{Состояние покоя}
    \lipsum[1-2]

\subsection{Геострофический поток}
    \lipsum[1-2]

\subsection{Поток Куэтта}
    \lipsum[1-2]

\subsection{Поток Пуазейля}
    \lipsum[1-2]

\section{Уравнения мелкой воды}
    \lipsum[1-2]

\section{Уравнения Буссинеска}
    \lipsum[1-2]

