\chapter*{ПРИМЕРЫ LaTex}

{\color{red}
    \[ \begin{bmatrix}
        0 && 1\\
        1 && 0
    \end{bmatrix}\]

\begin{table}[h]
    \centering
    \begin{tabular}{|c|c|} \hline 
         Column 1& Column 2\\ \hline \hline 
         1& 2\\ \hline 
         3& 4\\ \hline
    \end{tabular}
    \caption{This is a caption}
    \label{tab:my_label}
\end{table}

     \begin{align*}
         f'(x)&=\lim_{h\to 0}\frac{f(x+h)-f(x)}{h}\\
         &= \lim_{h\to 0}\frac{(x+h)^2-x^2}{h}\\
         &= \lim_{h\to 0}\frac{2xh+h^2}{h}\\
         &= \lim_{h\to 0}2x+h\\
         &=2x
     \end{align*}

% https://youtu.be/5jmIHOWpEg0
\begin{tikzpicture}
\begin{axis}[xmin=-10,ymin=-10,
xlabel=$x$, 
title={title титуль},
xmajorgrids=true,
ymajorgrids=true,
grid style = dashed]
\addplot[color=red, dashed, mark=*, samples=8]{x^2}node[right,pos=0.6]{$y={x^2}$};
\addplot[color=blue]{1-x^2};
\end{axis}
\end{tikzpicture}

}

\begin{boxA}
В приведенных выше сведениях из векторного анализа нам уже встретилось много известнык употребляемых в метеорологии  опереторов: градиента, дивергенции, индивидуальной производной. Теперь, используя эти понятия, обратимся к кинематике течений жидкости, то есть не будем рассматривать причин, порождавших вижение (поле скорости), что является предметом динамики, а остановимся лишь на характеристиках образовавшегося каким-то образом потока.
\end{boxA}

\begin{boxB}
В приведенных выше сведениях из векторного анализа нам уже встретилось много известнык употребляемых в метеорологии  опереторов: градиента, дивергенции, индивидуальной производной. Теперь, используя эти понятия, обратимся к кинематике течений жидкости, то есть не будем рассматривать причин, порождавших вижение (поле скорости), что является предметом динамики, а остановимся лишь на характеристиках образовавшегося каким-то образом потока.
\end{boxB}

\begin{boxC}
В приведенных выше сведениях из векторного анализа нам уже встретилось много известнык употребляемых в метеорологии  опереторов: градиента, дивергенции, индивидуальной производной. Теперь, используя эти понятия, обратимся к кинематике течений жидкости, то есть не будем рассматривать причин, порождавших вижение (поле скорости), что является предметом динамики, а остановимся лишь на характеристиках образовавшегося каким-то образом потока.
\end{boxC}

\begin{boxD}
В приведенных выше сведениях из векторного анализа нам уже встретилось много известнык употребляемых в метеорологии  опереторов: градиента, дивергенции, индивидуальной производной. Теперь, используя эти понятия, обратимся к кинематике течений жидкости, то есть не будем рассматривать причин, порождавших вижение (поле скорости), что является предметом динамики, а остановимся лишь на характеристиках образовавшегося каким-то образом потока.
\end{boxD}

\begin{boxE}
В приведенных выше сведениях из векторного анализа нам уже встретилось много известнык употребляемых в метеорологии  опереторов: градиента, дивергенции, индивидуальной производной. Теперь, используя эти понятия, обратимся к кинематике течений жидкости, то есть не будем рассматривать причин, порождавших вижение (поле скорости), что является предметом динамики, а остановимся лишь на характеристиках образовавшегося каким-то образом потока.
\end{boxE}

\begin{boxF}
В приведенных выше сведениях из векторного анализа нам уже встретилось много известнык употребляемых в метеорологии  опереторов: градиента, дивергенции, индивидуальной производной. Теперь, используя эти понятия, обратимся к кинематике течений жидкости, то есть не будем рассматривать причин, порождавших вижение (поле скорости), что является предметом динамики, а остановимся лишь на характеристиках образовавшегося каким-то образом потока.
\end{boxF}

\begin{boxG}
В приведенных выше сведениях из векторного анализа нам уже встретилось много известнык употребляемых в метеорологии  опереторов: градиента, дивергенции, индивидуальной производной. Теперь, используя эти понятия, обратимся к кинематике течений жидкости, то есть не будем рассматривать причин, порождавших вижение (поле скорости), что является предметом динамики, а остановимся лишь на характеристиках образовавшегося каким-то образом потока.
\end{boxG}

\begin{boxH}
В приведенных выше сведениях из векторного анализа нам уже встретилось много известнык употребляемых в метеорологии  опереторов: градиента, дивергенции, индивидуальной производной. Теперь, используя эти понятия, обратимся к кинематике течений жидкости, то есть не будем рассматривать причин, порождавших вижение (поле скорости), что является предметом динамики, а остановимся лишь на характеристиках образовавшегося каким-то образом потока.
\end{boxH}

\begin{boxI}
В приведенных выше сведениях из векторного анализа нам уже встретилось много известнык употребляемых в метеорологии  опереторов: градиента, дивергенции, индивидуальной производной. Теперь, используя эти понятия, обратимся к кинематике течений жидкости, то есть не будем рассматривать причин, порождавших вижение (поле скорости), что является предметом динамики, а остановимся лишь на характеристиках образовавшегося каким-то образом потока.
\end{boxI}

Test $\pd{x}{z}$ vs $\td{u}{t}$.