\chapter{Некоторые сведения из кинематики жидкости}

В приведенных выше сведениях из векторного анализа нам уже встретилось много известных употребляемых в метеорологии  операторов: градиента, дивергенции, индивидуальной производной. Теперь, используя эти понятия, обратимся к кинематике течений жидкости, то есть не будем рассматривать причин, порождавших движение (поле скорости), что является предметом динамики, а остановимся лишь на характеристиках образовавшегося каким-то образом потока.

В этой небольшой главе мы ограничимся лишь некоторыми определениями, которые будут нам встречаться в дальнейшем уже при получении динамики потока.

Из курса гидромеханики известно, что элементарные перемещения любой точки жидкой частицы можно рассмотреть как геометрическую сумму трех перемещений: поступательного, вращательного и деформационного.

Поток, в котором дивергенция равна нулю:
\begin{equation}
\nabla \cdot \vec{V} = 0,
\end{equation}
называется соленоидальным (несжимаемым) или бездивергентным.



Если мы предполагаем, что течения являются соленоидальными, а вертикальная скорость равна нулю или постоянна во всем слое, то оператор дивергенции становится
\begin{equation}
\nabla_h \cdot \vec{V} = \frac{\partial v_x}{\partial x} + \frac{\partial v_y}{\partial y} = 0
\end{equation}
Или, иначе 
\begin{equation}
\frac{\partial v_y}{\partial y} = \frac{\partial (-v_x)}{\partial x}
\end{equation}

Введем теперь понятие функции тока $\Psi$ и посмотрим, как получается выражение для этой функции.


